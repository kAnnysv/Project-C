\documentclass[a4paper,12pt]{article}

\usepackage{cmap}
\usepackage[T2A]{fontenc}
\usepackage[utf8]{inputenc}
\usepackage[english,russian]{babel}
\usepackage{graphicx}
\graphicspath{ {./image/} }

\author{brigidro}
\title{SmartCalc v1.0}

\begin{document}
	\maketitle
	\section{Основные функции}
		Арифметические операторы

\includegraphics[scale = 0.3]{pearme.png}


	\begin{itemize}

\itemОбласть определения и область значения функций ограничиваются по крайней мере числами от -1000000 до 1000000

\itemПроверяемая точность дробной части - минимум 7 знаков после запятой
\itemУ пользователя должна быть возможность ввода до 255 символов
\end{itemize}

	\section{Работа с калькулятором}
	\includegraphics[scale = 0.45]{clll.png}

		\begin{itemize}
\item Можно вводить значение переменной  X    в поле его ввода
\item Выражение для расчёта вводятся как c клавиатуры, так и с помощью кнопок приложения

	\end{itemize}

	\section{Работа с графиком}


	\begin{itemize}

		\item График можно строить линиями или точками (например, чтобы увидеть асимптоты)

		\item Если задать значения границ больше краевых значений, то значение устанавливается по умолчанию
	\end{itemize}

\end{document}
